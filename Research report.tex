\documentclass{letter}

%%%%%%%%%% Start TeXmacs macros
\newcommand{\section}[1]{\medskip\bigskip

\noindent\textbf{\LARGE #1}}
\newenvironment{tmparmod}[3]{\begin{list}{}{\setlength{\topsep}{0pt}\setlength{\leftmargin}{#1}\setlength{\rightmargin}{#2}\setlength{\parindent}{#3}\setlength{\listparindent}{\parindent}\setlength{\itemindent}{\parindent}\setlength{\parsep}{\parskip}} \item[]}{\end{list}}
%%%%%%%%%% End TeXmacs macros

\begin{document}

\title{Research Report}\author{John Mooney}\maketitle

\begin{center}
  \begin{tmparmod}{0pt}{0pt}{1em}
    \begin{tmparmod}{0pt}{2cm}{0pt}
      \begin{tmparmod}{2cm}{0pt}{0pt}
        A semi-chronological record of research in the Quantum Information
        Experimentation lab at Leeds University, including both the formal
        (hypothesis - test - conclude) and qualitative (personal experience
        and skills gained, possible sociological implications, etc) aspects. 
      \end{tmparmod}
    \end{tmparmod}
  \end{tmparmod}
\end{center}





{\tableofcontents}

{\newpage}

\section{Introduction}

The overriding aim this summer, was to produce something of scientific merit
involving the Rapman 3D printer we built . The Rapman is a commercial kit
based on the open source printer; The Reprap (Darwin). Thanks to the open
source nature of this project, may hundreds of individuals have collaborated
to produce a working 3D printer with print quality that rivals commercial
versions which can cost >{\pounds}10,000. The cultural impact of personal
manufacturing is speculated to be huge  especially when the technology used to
achieve it has on its side the development power of the GPL{\nobreak}.

\#any other points to make? replication, FDM process(belowparagraph), (abs
pla, hdpe),\#

The principal area for improving 3DP technology is software.\#have i given a
good argument?\# Though cheaper, higher resolution mechanics, refined
electronics design (capable of handling more intelligent software) and perhaps
better use of sensors for feedback, are all important. Currently a CAD file
(.stl format) is fed into a program which converts it into layers and then an
extruder head path, outputting machine code, G-Code (a list of coordinates and
other instructions for the printer to follow).

The generation of this G-Code requires knowledge of the printed polymers
physical properties, the most basic being the melting temperature of the
polymer. If these properties are known then polymer can be placed exactly
where it is needed, by compensating for visco-elastic effects and shrinkage on
cooling (and the different (position dependent) rates at which this happens,
leading to internal stresses and macro-scale warping), giving improved output
predictability and quality. Also, and this is the main focus of our research,
the filament can be deposited to favour the desired mechanical properties of
the final object, for example; increasing the density of polymer infill in
areas where tensile, compression or rotational stresses are expected to be
loaded.

Currently the only environmental feedbacks are nozzle and print bed
temperature (also extruder position (from the stepper motors), but this is not
in need of improvement), and volume of polymer filament injected into the
nozzle. The information from these sensors is all that should be needed to
have full knowledge of the polymers state, as the internal nozzle pressure,
outward flow rate, cooling time etc can all be calculated. Also, temperature
control is currently handled by a bang-bang feedback system; on if nozzle temp
below set value, off if above. This leads to large nozzle temperature
oscillations, which is not desirable because this temperature is used to infer
polymer state. Once this is replaced by a Proportional Integral Derivative
controller we can improve the software.

\#\#Theoretically the best printing would be achieved by using a polymer
physicists brain as the control electronics and G-Code writer. To clarify,\#\#

Interferometry, acoustic properties of plastic mounts

\section{Acoustic Properties}

--Printing mirror mounts, why; cheaper, custom shapes, heights etc, potential
novel properties,

ability to 'tune' a mount















\end{document}
